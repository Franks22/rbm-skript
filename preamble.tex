\documentclass[11pt]{article}
%
\usepackage[hyperref,dvipsnames]{xcolor}
\usepackage{hyperref}
\definecolor{darkblue}{rgb}{0,0,.5}                                                                                                 
\definecolor{black}{rgb}{0,0,0}                                                  
                                                                                 
\hypersetup{                                                                     
	breaklinks=true,                                                             
	bookmarksnumbered=true,                                                      
	bookmarksopen=true,                                                          
	bookmarksopenlevel=1,                                                        
	breaklinks=true,                                                             
	colorlinks=true,                                                             
	pdfstartview=Fit,                                                            
	pdfpagelayout=SinglePage,                                                    
	%                                                                            
	filecolor=darkblue,                                                          
	urlcolor=darkblue,                                                           
	linkcolor=black,                                                             
	citecolor=black                                                              
}

\usepackage{geometry}
\geometry{a4paper}
\usepackage{amssymb}
\usepackage{amsmath}
\usepackage{amsthm}
\usepackage[no-math]{fontspec}
\defaultfontfeatures{Mapping=tex-text}
\usepackage{polyglossia}
\setmainlanguage[spelling=new,babelshorthands=true]{german}
\usepackage{xltxtra}
\usepackage{graphicx}
\usepackage{float}
\usepackage{floatflt}
\usepackage{icomma}
%\usepackage{comment}
\usepackage{caption}
\usepackage{enumerate}
\usepackage{array}
\usepackage{tabularx}
\usepackage{cleveref}
\usepackage{accents}

\newcommand{\ubar}[1]{\underaccent{\bar}{#1}}
%
%\newcommand{\eq}[1]{\begin{align*}#1\end{align*}}
%\newcommand{\eqt}[1]{\begin{align}#1\end{align}}
%\newcommand{\mat}[1]{\left (\begin{matrix}#1\end{matrix}\right )}
\newcommand{\pd}[1]{\frac{\partial}{\partial {#1}}}
%%\newcommand{\qed}{\begin{flushright}$\square$\end{flushright}}
%\newcommand{\qed}{\hfill $\square$}
\newcommand{\dotp}[2]{\langle #1, #2 \rangle}
\newcommand{\pdot}{\,\cdot\,}
\newcommand{\norm}[1]{\|#1\|}
%\newcommand{\mrm}[1]{\mathrm{#1}}
%\newcommand{\satz}[2]{\paragraph{Satz #1:}#2}
%\newcommand{\bew}[1]{\paragraph{Beweis:}#1}
%\newcommand{\defn}[2]{\paragraph{Definition #1:}#2}
%\newcommand{\bsp}[1]{\paragraph{Beispiel:}#1}
%\newcommand{\bem}[1]{\paragraph{Bemerkung:}#1}
%\newcommand{\kor}[2]{\paragraph{Korollar #1:}#2}
%\newcommand{\lemma}[2]{\paragraph{Lemma #1:}#2}
\renewcommand{\O}{\mathcal O}
%\newcolumntype{M}{>{$}c<{$}}
%%\renewcommand*\arraystretch{1.5}
%\newcommand{\C}{\operatorname{C}}
%\renewcommand{\epsilon}{\varepsilon}
\let\origphi\phi
\let\phi\varphi
\let\varphi\origphi
%\renewcommand{\phi}{\varphi}
%\renewcommand{\Re}{\operatorname{Re}}
%\renewcommand{\Im}{\operatorname{Im}}
%\renewcommand{\d}{\; \mathrm{d}}
\newcommand{\D}{\mathrm{D}}
%\newcommand{\supp}{\operatorname{supp}}
%\newcommand{\diag}{\operatorname{diag}}
%\newcommand{\Bild}{\operatorname{Bild}}
%\newcommand{\Span}{\operatorname{Span}}
%\newcommand{\Kern}{\operatorname{Kern}}
%\newcommand{\pdef}[1]{\left\{\begin{array}{lc}#1\end{array}\right.}
\newcommand{\R}{\mathbb{R}}
\newcommand{\N}{\mathbb{N}}
\newcommand{\p}{\mathcal{P}}
\newcommand{\LM}{\mathcal{M}}
\newcommand{\op}{\operatorname}
\newcommand{\prob}[1][\mu]{\big(P(#1)\big)}
\newcommand{\rprob}[1][\mu]{\big(P_N(#1)\big)}
\newcommand{\du}{\mathrm{du}}
\newcommand{\dprob}[1][\mu]{\big(P^\du(#1)\big)}
\newcommand{\pdprob}[1][\mu]{\big(P_N'(#1)\big)}
\newcommand{\refprob}[1][\mu]{\big(\hat P(#1)\big)}
\newcommand{\set}[1]{\left\{#1\right\}}
\newcommand{\spn}{\operatorname{span}}
\newcommand{\seq}[1]{\left(#1\right)}
\newcommand{\dist}{\operatorname{dist}}

\newcommand{\beginwithlist}{\leavevmode \vspace{-\baselineskip}}
\newcommand{\beginwithlistbem}{\leavevmode}
\newcommand{\beginwithlistbew}{\leavevmode}

\numberwithin{equation}{section}
\renewcommand{\labelenumi}{\roman{enumi}\,)}

\newtheoremstyle{mythmstyle}
	{\baselineskip}%		space above
	{\baselineskip}%		space below
	{}%			body font
	{}%			indent
	{\bfseries}%			head font
	{}%			punctuation after head
	{\newline}%	space after head
	{}%			'theorem head spec'

\theoremstyle{mythmstyle}

\newtheorem{satz}{Satz}[section]
\newtheorem{defn}[satz]{Definition}
\newtheorem{lemma}[satz]{Lemma}
\newtheorem{kor}[satz]{Korollar}
\newtheorem{bsp}[satz]{Beispiel}

\theoremstyle{definition}

\newtheorem*{bem}{Bemerkung}
